\documentclass[a4paper]{scrreprt}
 
\usepackage[german]{babel}
\usepackage[utf8]{inputenc}
\usepackage[T1]{fontenc}
\usepackage{ae}
\usepackage[bookmarks,bookmarksnumbered]{hyperref}
 
\begin{document}
 
\title{Pflichtenheft}
\author{WatchPLB}
\maketitle
 
\begin{abstract}
%TODO Projektbeschreibung
\end{abstract}
 
\tableofcontents
 
\chapter{Zielbestimmung}
 
\section{Musskriterien}
\begin{itemize}
	\item Testumgebung realisieren
	\item PLB Registrieren und Zulassen

	\item Realisieren von Status-LEDs für folgende Informationen:
		\subitem GPS Signal Vorhanden
		\subitem Akkustand
		\subitem Notruf wird abgesetzt

	\item Zusätzlich Vibrationsalarm wenn Notruf abgesetzt wird

	\item Verzögerung von einer Minute bis zum Senden des Notrufes um Notruf abbrechen zu können

	\item Absetzen des Notrufes über 2 Tasten die gleichzeitig für mindestens 10 Sekunden gedrückt werden müssen.
	
	\item Realisieren einer Ladeschaltung
\end{itemize}

 
\section{Kannkriterien}
\begin{itemize}
	\item Einschalten des Geräts erst bei Tastendruck
	\item Wasserdicht nach mindestens IP65
	\item Formfaktor einer Uhr
	\item Android-App
		\subitem Anbindung über Bluetooth oder Bluetooth LE
		\subitem Anzeigen von Akkustand
		\subitem Positionsvergleich zwischen PLB und Android Gerät
\end{itemize}
 
\section{Abgrenzungskriterien}
\begin{itemize}
	\item Peilsender
	\item Einsatz unter Wasser
	\item Einsatz in Schluchten
	\item Einsatz im Schnee
\end{itemize}
 
\chapter{Einsatz}
Der geplante Einsatz des Systems ist die Grundlage für Benutzungsoberfläche und
Qualitätsanforderungen.
 
\section{Anwendungsbereiche}
Ein Pflichtenheft wird bspw. in einer IT-Abteilung genutzt.
 
\section{Zielgruppen}
Die Zielgruppe besteht also aus Informatikern, die mit der Projektplanung
beauftragt wurden.
 
\section{Betriebsbedingungen}
Betriebsbedingungen: Die Betriebsbedingungen spezifiziert die physikalische
Umgebung des Systems, die tägliche Betriebszeit, und ob das System ständiger
Beobachtung durch Bediener ausgesetzt ist, oder ein unbeaufsichtigter Betrieb
beabsichtigt ist.
 
\chapter{Umgebung}
 
\section{Software}
Software: Gibt an, welche Software zum Betrieb vorhanden sein muss. Eine
Aufteilung für Server und Client ist ggf. sinnvoll. Weiterhin sind unbedingt die
kleinsten benötigten Versionsnummern anzugeben.
 
\section{Hardware}
Hardware: Hardware-Anforderungen des Systems.
 
\section{Orgware}
Orgware: Angabe der organisatorische Rahmenbedingungen, die vor Projektstart
erfüllt sein müssen.
 
\chapter{Funktionalität}
Funktionalität: Spezifikation der einzelnen Produktfunktionen mit genauer und
detaillierter Beschreibung.
 
\begin{itemize}
  \item Typische Arbeitsabläufe
  \item Keine Angabe von typischen Verwaltungsfunktionen (CRUD \footnote{Create,
Read, Update, Delete}
\end{itemize}
 
\chapter{Daten}
Daten: Angabe der Daten, die langfristig aus Benutzersicht zu speichern sind.
 
\chapter{Leistungen}
Leistungen: Anforderungen bezüglich Zeit und Genauigkeit
 
\chapter{Benutzungsoberfläche}
Benutzungsoberfläche: grundlegende Anforderungen, Zugriffsrechte
 
\begin{figure}[ht]
  \centering
  \rule{8cm}{6cm}
  \caption{Dies könnte ein Bild der Benutzungsoberfläche sein}
\end{figure}
 
\chapter{Qualitätsziele}
Qualiätsziele: Allgemeine Ziele sind meistens Änderbarkeit und Wartbarkeit.
Ziele sollten jedoch grundsätzlich messbar, spezifisch und relevant sein.
 
\chapter{Ergänzungen}
Hier ist Platz für nicht im Pflichtenheft abgedeckte Themengebiete oder ein
Glossar.
 
\listoffigures
 
\end{document}