%%%%%%%%%%%%%%%%%%%%%%%%%%%%%%%%%%%%%%%%%%%%%%%%%%%%%%%%%%%%%%%%%%%%%%%%%%%%%%%%%%%%%%%%%%%%%%
%% @file		Specification.tex
%% @author		Christopher Reiterer
%% @date		18.03.2015
%% @brief		Specification template
%% @remarks		-
%% @version		1.0
%%%%%%%%%%%%%%%%%%%%%%%%%%%%%%%%%%%%%%%%%%%%%%%%%%%%%%%%%%%%%%%%%%%%%%%%%%%%%%%%%%%%%%%%%%%%%%

%%%%%%%%%%%%%%%%%%%%%%%%%%%%%%%%%%%%%%%%%%%%%%%%%%%%%%%%%%%%%%%%%%%%%%%%%%%%%%%%%%%%%%%%%%%%%%
%% @file		Definitions.tex
%% @author		Christopher Reiterer
%% @date		18.03.2015
%% @brief		Definitions for the specification template
%% @remarks		-
%% @version		1.0
%%%%%%%%%%%%%%%%%%%%%%%%%%%%%%%%%%%%%%%%%%%%%%%%%%%%%%%%%%%%%%%%%%%%%%%%%%%%%%%%%%%%%%%%%%%%%%

\documentclass[11pt,a4paper]{scrartcl}

\usepackage[a4paper, left=2cm, right=2cm, bottom=1.5cm, top=1.3cm, includeheadfoot]{geometry}
\usepackage[ngerman]{babel}
\usepackage[utf8]{inputenc} % comment this if you uncomment utf8x
%\usepackage[utf8x]{inputenc} % uncomment this if there are problems with 'ä', 'ü', 'ö'
\usepackage{ucs}
\usepackage[usenames,dvipsnames,table]{xcolor}
\usepackage[fleqn]{amsmath}
\usepackage{amsfonts}
\usepackage{amssymb}
\usepackage{color}
\usepackage{listings}
\usepackage{hyperref}
\usepackage{amsfonts}
\usepackage{listings}
\usepackage[pdftex]{graphicx}
\usepackage[T1]{fontenc}
\usepackage{titlesec}
\usepackage{fancyhdr}
\usepackage{lastpage}
\usepackage{courier}
\usepackage{ctable}

%Definitions of header and footer
\pagestyle{fancy}
\fancyhf{}
\renewcommand{\headrulewidth}{0pt}
\renewcommand{\footrulewidth}{0.4pt}
\lfoot{WatchPLB}
\cfoot{Projekt Pflichtenheft}
\rfoot{\thepage}


%color definitions
\definecolor{lightgray}{rgb}{0.85,0.85,0.85}

\addtokomafont{section}{\mysection}
\newcommand{\mysection}[1]{%
    \centering
    \setlength{\fboxsep}{0cm}%already boxed
    \colorbox{lightgray!80}{%
        \begin{minipage}{\linewidth}%
            \vspace*{3pt}%Space before
            #1
            \vspace*{0pt}%Space after
        \end{minipage}%
    }}
	
\newcommand{\csection}[1]{\section[#1]{\centering #1}} 		%command for centered section

\newcommand{\HRule}{\rule{\linewidth}{0.3mm}}				%horizontal line definition
\newcommand\VRule[1][\arrayrulewidth]{\vrule width #1}		%vertical line difinition

\begin{document}

%%%%%%%%%%%%%%%%%%%%%%%%%%%%%%%%%%%%%%%%%%%%%%%%%%%%%%%%%%%%%%%%%%%%%%%%%%%%%%%%%%%%%%%%%%%%%%
%% @file		TitlePage.tex
%% @author		Christopher Reiterer
%% @date		18.03.2015
%% @brief		title page for the specification template
%% @remarks		-
%% @version		1.0
%%%%%%%%%%%%%%%%%%%%%%%%%%%%%%%%%%%%%%%%%%%%%%%%%%%%%%%%%%%%%%%%%%%%%%%%%%%%%%%%%%%%%%%%%%%%%%

\begin{titlepage}
\begin{flushright}		%right alignment
~
\\[4.2cm]

\textbf{\Huge WatchPLB}
\HRule \\[0.3cm]
{\LARGE Projekt Pflichtenheft}
~
\\[5.7cm]
\textbf{\LARGE ~} \\
\textbf{\LARGE ~}
\\[7.5cm]

\begin{tabular}{r@{\hspace{2cm}}  r}

\textbf{Version Nr.:} & 01.00.00 \\[0.2cm]
\textbf{Autor:} & Egger, Neuhofer, Götzinger, Hick, Kanzler, Svyrydova \\[0.2cm]
\textbf{Letztes Revisionsdatum:} & 18.04.2018 \\[0.2cm]
\textbf{Dokument Status:} & Final \\[0.2cm]
\textbf{Datei:} & {\large Pflichtenheft}

\end{tabular}


\end{flushright}
\end{titlepage}

\setcounter{secnumdepth}{0}		%disable numbering for this section
\csection{Revisions}

\begin{tabular}{!{\VRule[1.7pt]} p{1.9cm} | p{1.9cm} | p{7cm} | p{4.5cm} !{\VRule[1.7pt]}}

\specialrule{1.9pt}{0pt}{0pt} 
\cellcolor{lightgray!130}\textbf{Version} & 
\cellcolor{lightgray!130}\textbf{Datum} & 
\cellcolor{lightgray!130}\textbf{Kommentar} & 
\cellcolor{lightgray!130}\textbf{Autor} \\
\hline
00.00.00 & 
dd.mm.yy & 
Information about the revision. This table does not need to be filled in whenever a document is touched, only when the \textbf{version is being upgraded.} & 
Name \\
\hline
&&& \\
\specialrule{1.7pt}{0pt}{0pt} 

\end{tabular}
\\[0.7cm]
\noindent \{ The paragraphs written in the “Comment” style are for explanations and help for the document writer and should be removed in the “real” document. \\[-0.3cm]

\noindent This template can be used to create Software Requirements Specifications that conform to IEEE Standard 830-1993. \\[-0.3cm]

\noindent An SRS is a tool for capturing requirements on a project; it is the epitome of “plain language requirements”. Although an SRS is designed to stand on its own, most projects will employ additional tools for capturing requirements. An SRS may be ancillary or unnecessary on many projects, but a partial or lightweight one may be quite useful even if other techniques are being used to capture the bulk of the requirements. \\[-0.3cm]

\noindent Consider using an SRS when: \\[0.1cm]
Modeling techniques need to be augmented \\[0.1cm]
Plain language is the best mechanism for capturing system behavior \\[0.1cm]
Requirements must be traceable \\[0.1cm]
Required by regulations\}

\newpage
\renewcommand*\contentsname{\hfill Inhaltsverzeichnis \hfill} %center the title of the table of contents
\tableofcontents
\newpage

\setcounter{secnumdepth}{3} 		%enable section numbering
\section{Einleitung}
\subsection{Sinn und Zweck}
Im Rahmen der Projektarbeit "WatchPLB"\ an der FH Hagenberg erstellt das zugehörige Projektteam sowohl die nötige Soft-, als auch Hardware für ein Satellitengestütztes, persönliches Notrufsystem, genannt PLB. Das Projektteam besteht aus Tobias Egger, Michael Neuhofer, Samuel Hick, Olga Svyrydova, Eva- Maria Kanzler und Paul Götzinger. Als Projektvorstand dient Florian Eibensteiner und als Projektleiter Tobias Egger. Die hier festgelegten Anforderung und Bestimmung sind verpflichtend Einzuhalten und zur Produktreife zu bringen. Alle Projektmitglieder verpflichten sich dazu, diese Anforderungen einzuhalten.
\subsection{Umfang und Anwendungsbereich}
Es wird ein tragbares PLB entwickelt, welches sich an Wasser- und Extremsportler richtet. Es muss ein, durch GPS- Daten gestützter, COSPAS- SARSAT- Notruf über die Notruffrequenz 406 MHz abgesetzt werden können. Diesem Notruf müssen alle benötigten Positionsdaten, sowie zumindest noch die persönliche Kennung gemäß der Registrierung mitgegeben werden, damit das System in Europa funktionsfähig ist. Eine Erweiterung des Systems stellt das integrierte Bluetooth Modul dar, welches die Funktionalität um eine Ausgabe der Positions- und Vitaldaten des Systems für eine spätere Verwendung in Apps oder dergleichen verwendet. Zudem soll noch eine Android App implementiert werden, welche es ermöglicht, eben diese Vital- und Positionsdaten auszulesen und mit den Positionsdaten des Smartphones zu vergleichen. Dabei sollen alle Daten, sowohl die des Smartphones, als auch die des PLBs, automatisch am Smartphone mitgeloggt werden. Der Grund hierfür ist eine optionale, spätere Auswertung dieser Daten und einer damit verbunden Fehleranalyse.
\subsection{Definitionen und Abkürzungen}
\begin{tabular}{| p{2cm} |p{3cm}| p{11cm} |}
	
	\hline
	\textbf{Abkür-} \newline \textbf{zung} & 
	\textbf{Bedeutung} & \textbf{Definition}\\
	\hline
	
	PLB & 
	Personal Locator Beacon &
	 Tragbares, für die persönliche Nutzung gedachtes, Gerät zum Absetzen eines Notrufs.\\
	\hline
	
	Button & 
	Knopf &
	Ein Knopf \\
	\hline
	
	GPIO & 
	General-purpose input/output &
	Hardwareinterface eines Prozessors zur Ein- oder Ausgabe von Daten in oder aus dem Prozessor.\\ 
	\hline
	
	SPI & 
	Serial Peripheral Interface &
	Serieller Datenbus für elektronische Geräte \\
	\hline
	
	UART &
	Universal Asynchronous Receiver Transmitter &
	Transmitter zur asynchronen, seriellen Datenübermittlung \\
	\hline
	
	I2C &
	Inter-Integrated Circuit &
	Serieller Datenbus zur Kommunikation mit verschiedenen Geräten\\
	\hline
	
	COPSAS- SARSAT & 
	Satellitennetzwerk &
	Internationales Satellitennetzwerk zur Aufspürung und Weiterleitung von Notrufen\\
	\hline
	
	OQPSK & 
	Offset quadrature phase-shift keying &
	Funkmodulation	\\
	\hline
	
	IDE & Integrated development environment
	 & Programmierumgebung für das einfache Programmieren, Übersetzen und zur Fehlersuche.
	\\
	\hline
	
	Debugger & 
	 & Tool zur Fehlersuche im Programm
	\\
	\hline
	
	Compiler & 
	Übersetzer &
	Programm zum Übersetzen des Programmcodes in ein, für den Prozessor, nutzbares Format
	\\
	\hline
	
	Makefile & 
	Erstellungsskript
	& Eine Art Skript, um den Erstellungsprozess zu automatisieren.
	\\
	\hline

	HF & 
	Hochfrequenz
	& 
	\\
	\hline	

	Modulation & 
	& Veränderung eines Trägersignals.
	\\
	\hline
	
\end{tabular}
\subsection{Prioritäten}

\begin{tabular}{| p{3cm} | p{2cm} | p{10cm} |}

\hline
\textbf{Priorität} & 
\textbf{Abkür-} \newline \textbf{zung} & 
\textbf{Bedeutung} \\
\hline

Muss & 
M & 
Diese Anforderung ist unabdingbar und notwendig für das korrekte Funktionieren der Software; sie muss realisiert werden. \\
\hline

Soll &
S &
Diese Anforderung ist nicht unabdingbar ihre Realisierung trägt zur wesentlichen Verbesserung der Software bei. Sie soll wenn möglich realisiert werden. \\
\hline

Wunsch &
W &
Diese Anforderung trägt zur Verbesserung der Software bei, ist jedoch nicht unbedingt notwendig. Es wäre aus wünschenswert, wenn die Anforderung realisiert würde. \\
\hline

\end{tabular}

\begin{tabular}{| p{1.5cm} | p{13.5cm} |}
	
	\hline
	\textbf{Priori-} \newline \textbf{tät} & 
	\textbf{Bedeutung} \\
	\hline
	
	M & 
	Bau eines 406 MHz Senders, welcher genug Leistung hat, um von einem Satelliten Zuverlässig empfangen zu werden. (5Watt) \\
	\hline
	
	M &
	Einbau eines Bluetooth LE- Moduls zur späteren Weiterverwendung in der optionalen App \\
	\hline
	
	M &
	Herstellung eines Prototypen, mit dem sich erfolgreich ein Notruf absetzen lässt  \\
	\hline
	
	M &
    Einbau und Programmierung eines GPS- Moduls und Weitergabe dieser Daten an den Notruf  \\
	\hline
	
	M &
	Erstellung eines PCBs und Integration aller elektronischen Bestandteile außer Akku, LEDs und Buttons  \\
	\hline
	
	M &
	Verbau eines Vibrationsmotors um haptisches Feedback über den Status des Notrufes an den Nutzer weiter zu geben  \\
	\hline
	
	M &
	Nutzung und Verbau zweier Buttons, welche gleichzeitig für 5 Sekunden gedrückt werden müssen, um den Notruf auszulösen. Mithilfe eines gleichzeitigen 10 sekündigen Tastendrucks kann der Notruf abgebrochen werden \\
	\hline
	
	M &
	Nutzung und Verbau eines weiteren Buttons zur Aktivierung des Bluetooth LE Moduls. Sowohl zum ein, als auch ausschalten, muss der Button mindestens 2 Sekunden gedrückt werden \\
	\hline
	
	M &
	Verbau eines mehrfärbigen LED- Histogrammes mit 5 LEDs zur Angabe des Akkustandes(Rot, Orange, Gelb, Grün, Grün). Der Akkustand wird in 20 Prozent Schritten angegeben. \\
	\hline
	
	M & 
	Verbau einer RGB- LED, welche grün leuchtet, wenn der Notruf ausgelöst werden kann, gelb wenn er ausgelöst wurde, rot wenn der Notruf aktiv ist und weiß, wenn er ausgelöst aber abgebrochen wurde \\
	\hline
	
	M & 
	Verbau einer Blauen LED, welche wenn aktiviert indiziert, dass das Bluetooth Modul eingeschalten wurde.\\
	\hline
	
	M & 
	Verbau einer Micro- B USB Ladeschaltung direkt auf dem PCB\\
	\hline
	
	M & 
	Verbau eines Akkus, welcher mindestens 2 Stunden Betriebszeit bei abgesetzten Notruf gewährleistet\\
	\hline
\end{tabular}
\newpage
\begin{tabular}{| p{1.5cm} | p{13.5cm} |}
	
	\hline
	\textbf{Priori-} \newline \textbf{tät} & 
	\textbf{Bedeutung} \\
	\hline
	
	W & 
	Bau eines Gehäuses, welches dem Gerät Schutz nach IP55 gibt
	Zudem soll es das Gehäuse das Gerät so stark schützen, dass das Gerät einen Sturz aus 1.5m problemlos übersteht.
	In das Gehäuse sollen die LEDs und Buttons integriert werden. Die Ladeschaltung soll mithilfe eines Stöpsels versiegelt werden.\\
	\hline
	S &
	Es soll eine minimale Bluetooth LE- API zum auslesen folgender Daten bereitgestellt werden: Akkustand, GPS- Postion, Ob der Notruf Aktiv ist, Ob der Notruf angefordert wurde, ob der Notruf angefordert wurde, wie oft der Notruf abgesetzt wurde. Wie lange der Notruf aktiv ist, wie oft er abgesetzt oder abgebrochen wurde, und Fehlermeldungen in Form kurzer Strings.\\
	\hline
	S &
	Erstellung einer Android 6 Smartphone App zur Ausgabe der über die oben beschriebenen API- Daten. Die App soll einen Tab haben, in welchem die GPS- Daten des Smartphones und des PLBs auf einer Karte eingezeichnet werden, und eines Tabs, der alle sonstigen Beschriebenen Daten in Textform untereinander Ausgibt. Zudem soll die APP alle Empfangenen Daten, sowie die eigene GPS Position bei Eingang der Daten mitloggen. Nützlich für eine spätere statistische Analyse\\
	\hline
	
	W & 
	Formfaktor einer Uhr. Das Betrifft sowohl das Gehäuse, als auch das Gerät. Antenne soll integriert sein.\\
	\hline
	
	S &
	Erstmaliges Einschalten des Geräts erst bei zwei sekündigen, einmaligen Tastendruck auf einen Button.\\
	\hline
	
	W & 
	Bau eines Funktionstüchtigen Homing Senders auf 121 und 243 MHz\\
	\hline
	
	W & 
	Zulassung des PLBs als PLB in Österreich.\\
	\hline
	
\end{tabular}


\newpage

\section{Allgemeine Beschreibung}
In this section, describe the general factors that affect the product and its requirements. This section should contain background information, not state specific requirements.

\subsection{Beziehung zu anderen Systemen}
This section should place the product in perspective with other related products. If the product is independent and self-contained, state that here. Otherwise, identify interfaces between the product and related systems.

\subsubsection{Systemschnittstellen}
List each system interface and identify the related functionality of the product.

\subsubsection{Benutzerschnittstellen}
Specify the logical characteristics of each interface between the software product and its users (e.g., required screen formats, report layouts, menu structures, or function keys). \\[-0.3cm]

\noindent Specify all the aspects of optimizing the interface with the person who must use the system (e.g., required functionality to provide long or short error messages). This could be a list of do’s and don’ts describing how the system will appear to the user.

\subsubsection{Hardwareschnittstellen}
Specify the logical characteristics of each interface between the software product and the hardware components of the system. This includes configuration characteristics (e.g., number of ports, instruction sets), what devices are to be supported, and protocols.

\subsubsection{Softwareschnittstellen}
Specify the use of other required software products (e.g., a DBMS or operating system), and interfaces with other application systems. \\[-0.3cm]

\noindent For each required software product, provide identification information including at least name, version number, and source. \\[-0.3cm]

\noindent For each interface, discuss the purpose of the interfacing software, and define the interface in terms of message format and content. For well-documented interfaces, simply provide a reference to the documentation. \\[-0.3cm]

\subsubsection{Kommunikationsschnittstellen}
Specify any interfaces to communications such as local area networks, etc.

\subsubsection{Speicher Bedingungen}
Specify any applicable characteristics and limits on RAM, disk space, etc.

\subsubsection{Operationen}
Specify any normal and special operations required by the user, including:
\begin{itemize}
	\item periods of interactive operations and periods of unattended operations
	\item data processing support functions
	\item backup and recovery operations
	\item etc.
\end{itemize}

\subsubsection{Installations- und Anpassungs-Anforderungen}
Define requirements for any data or initialization sequences that are specific to a given site, mission, or operational mode. Specify features that should be modified to adapt the software to a particular installation.

\subsection{Produkt Funktionen}
Provide a summary of the major functions that the software will perform.

\subsection{Benutzereigenschaften}
Describe the general characteristics of the intended users, including
\begin{itemize}
	\item educational level
	\item experience
	\item technical expertise
\end{itemize}

\subsection{Allgemeine Einschränkungen}
Describe any other items that will constrain the design options, including
\begin{itemize}
	\item reliability requirements
	\item criticality of the application
	\item safety and security considerations regulatory policies
	\item hardware limitations
	\item interfaces to other applications
	\item parallel operation
	\item audit functions
	\item control functions
	\item higher-order language requirements
	\item signal handshake protocols
	\item etc.
\end{itemize}

\subsection{Annahmen und Abhängigkeiten}
List factors that affect the requirements. These factors are not design constraints, but areas where future changes might drive change in the requirements.

\subsection{Realisierung der Anforderungen}
Identify any requirements that may be delayed until future versions of the system.

\newpage

\section{Externe Schnittstellen}
From this section on all software requirements should be described at a sufficient level of detail for designers to design a system satisfying the requirements and testers to verity that the system satisfies requirements. \\[-0.3cm]

\noindent The organization of the remainder of this sample document is derived from the A.5 Template of SRS Section 3 Organized by Feature shown in the Annex of Std 830-1993. For alternative organizational schemes by system mode, user class, object, stimulus, functional hierarchy, and combinations, see the standard. \\[-0.3cm]

\noindent Provide a detailed description of all inputs into and outputs from the software. This section should complement the interface descriptions under section 2.1 and should not repeat information there. Include both content and format as follows:
\begin{itemize}
	\item name of item
	\item description of purpose
	\item source of input or destination of output
	\item valid range, accuracy, and/or tolerance
	\item units of measure
	\item timing
	\item relationships to other inputs/outputs
	\item screen formats/organization
	\item window formats/organization
	\item data formats
	\item command formats
	\item end messages
\end{itemize}

\noindent These requirements may be organized in the following subsections. \\[0.1cm]
IMPORTANT: Number each item

\subsection{Benutzerschnittstellen}

\subsection{Hardwareschnittstellen}

\subsection{Softwareschnittstellen}

\subsection{Kommunikationsschnittstellen}

\newpage

\section{Funktionale Anforderungen}
This section describes the actual requested user functions of the product and is a refinement of section 2.4.1. These requirements may be listed as “system features” or as use cases, whichever is your preferred method. \\[-0.3cm]

\noindent Anyway the features as well as use cases should be structured according their granularity.
\begin{itemize}
	\item[--] few ''System Features`` containing functional requirements or
	\item[--] very few “summary level” use cases (only needed for large systems) containing more “user-goal level” use cases containing other “function level” use cases below (use a use case template for describing use cases)
	\item[] Alternative:
	\item[--] an alternative to using the use case granularity for structuring the document, is to group it into logical packages (use cases which belong ''somehow`` together)
\end{itemize}

\subsection{System Feature 1 / Use Case 1}
Repeat subsections at this level and below for each feature. \\[-0.3cm]

\noindent \textit{\large Functional Requirement 1} \\
Repeat subsections at this level and below for each associated functional requirement. \\[-0.3cm]

\noindent Each functional requirement may be described in natural language, pseudocode, or in four subsections as follows. Functional requirements include:
\begin{itemize}
	\item validity checks on inputs
	\item exact sequencing of operations
	\item responses to abnormal situations, including error handling and recovery
	\item effects of parameters
	\item relationships of inputs to outputs, including input/output sequences and formulas for input to output
conversion
\end{itemize}

\noindent \textit{\large Introduction} \\
\textit{\large Inputs} \\
\textit{\large Processing} \\
\textit{\large Outputs} \\

\newpage

\section{Andere Nicht-Funktionale Anforderungen}

\subsection{Leistungsanforderungen}
Specify static and dynamic numerical requirements placed on the software or on human interaction with the software. \\[-0.3cm]

\noindent Static numerical requirements may include the number of terminals to be supported, the number of simultaneous users to be supported, and the amount and type of information to be handled. \\[-0.3cm]

\noindent Dynamic numerical requirements may include the number of transactions and tasks and the amount of data to be processed within certain time period for both normal and peak workload conditions.

\subsection{Entwurfsbedingungen}
Specify requirements imposed by standards, hardware limitations, etc.

\subsection{Software Qualitätsanforderungen}
The following items provide a partial list of system attributes that can serve as requirements that should be objectively verified. \\[0.1cm]
Other possible options include scalability, portability, robustness, recoverability, etc.

\subsubsection{Zuverlässigkeit}

\subsubsection{Verfügbarkeit}

\subsubsection{Sicherheit}
Specify the factors that will protect the software from accidental or malicious access, misuse, or modification. These factors may include:
\begin{itemize}
	\item cryptography
	\item activity logging
	\item restrictions on intermodule communications
	\item data integrity checks
\end{itemize}

\subsubsection{Wartbarkeit / Adaptionsfähigkeit}
Specify attributes of the software that relate to ease of maintenance. These requirements may relate to modularity, complexity, or interface design. Requirements should not be placed here simply because they are thought to be good design practices.

\subsection{Datenbank Anforderungen}
Specify the requirements for any information that is to be placed into a database, including
\begin{itemize}
	\item types of information used by various functions
	\item frequency of use
	\item accessing capabilities
	\item data entities and their relationships
	\item integrity constraints
	\item data retention requirements
\end{itemize}

\subsection{Weitere Anforderungen}

\subsubsection{Kontrollfunktionen}

\subsubsection{Traces}

\subsubsection{Fehlerdokumentation}

\subsubsection{Support und Service}

\newpage

\section{Analyse Modell}
This chapter describes the complete conceptual model for the problem domain. It lists some must to have content (domain classes, domain model) and some possibly helpful modeling tools. Add/remove models as needed.

\subsection{Domänen Klassen}
For each class specify its name, its purpose in the domain and its properties.

\subsubsection{ClassName1}

\begin{tabular}{| p{2cm} | p{2cm} | p{9.5cm} |}

\hline
Zweck: &  
\multicolumn{2}{p{11.5cm} |}{} \\
\hline

Attribute: & 
PropName & 
Description \\
\hline

\end{tabular}

\subsubsection{ClassName2}

\subsection{System Interactions}
\{Specifiy the system operations, i.e. the interactions of actors with the system, for important use cases as system sequence diagrams, for example\}

\subsection{Konzeptuelles Modell}
\{Specifiy the conceptional class model (domain model)\}

\subsection{Datenmodell}

\subsection{Zustandsdiagramme}

\newpage

\setcounter{secnumdepth}{0}
\section{Anhang}

\subsection{Anhang A: Index}

\subsection{Anhang B: To be determined list}

\end{document}